\documentclass{resume}
\usepackage[left=0.75in,top=0.6in,right=0.75in,bottom=0.6in]{geometry}
\usepackage{amsmath}

\begin{document}

\begin{rSection}{Technical Skills}

    \begin{itemize}
        \item Languages/Tools: \textbf{Python}, \textbf{Java}, \textbf{C++}, SQL, R \textbar\ Git, Docker, \textbf{CUDA}, AWS, Apache Kafka, Kubernetes.
              % \item Languages/Tools: Python, C++, Java, JavaScript, HTML, CSS, R, SQL.
        \item Machine Learning Libraries: NumPy, SciPy, pandas, \textbf{TensorFlow}, PyTorch, Caffe, scikit-learn, OpenCV.
    \end{itemize}

\end{rSection}

\begin{rSection}{Work Experiences}

    \begin{rSubsection}{NVIDIA Corporation}{January 2022 \(-\) April 2022 \(\cdot\) 4 mos}
        {Computer Vision Engineer - Autonomous Vehicles}{Santa Clara, CA, United States (Remote)}
        \item Defined a clear \textbf{data exportation workflow} and enabled relevant teams to create clean datasets for model development
        and comparison between different versions.
        \item Implemented cyclical learning rate schedules, over/undersampling mechanisms, and refactored code for model definition
        to enable more robust fine-tuning process.
        \item Improved the \(\boldsymbol{F_{1}}\)\textbf{-score} of a traffic light \textbf{classification} model by around \textbf{1\%} on \textbf{end-to-end KPI} test sets
        by fine-tuning from thousands of training experiments.
        \item Improved training methodologies and \textbf{reduced training time} from around 20h to around 3h.
        \item Fixed memory, latency, and performance \textbf{tests} for multiple classifier nodes on different platforms and generated \textbf{reports} for other teams to review.
    \end{rSubsection}

    \begin{rSubsection}{MIND Technologies Inc}{May 2021 \(-\) August 2021 \(\cdot\) 4 mos}
        {Machine Learning Engineer}{The Woodlands, TX, United States (Remote)}
        \item Generated \textbf{synthetic data} of lobster pots to \textbf{pretrain} the RetinaNet model.
        \item \textbf{Transferred} a RetinaNet \textbf{object detection} model
        from the COCO 2017 dataset to sonar signals of underwater lobster pots.
        \item Fine-tuned the feature pyramid architecture and achieved \textbf{near 1.0 confidence} on synthetic data.
        \item Deployed the model onto \textbf{Google Edge TPU} using TensorFlow Lite
        and \textbf{NVIDIA Jetson Nano} using TensorRT,
        and profiled the usages.
    \end{rSubsection}

    \begin{rSubsection}{University of Waterloo}{May 2022 \(-\) August 2022 \(\cdot\) 4 mos}
        {Undergraduate Research Assistant}{Waterloo, ON, Canada}
        \item Proposed and proved a \textbf{closed form formula} for \textbf{projection} onto intersection of a closed affine subspace and a hyperplane in a Hilbert space.
        \item Implemented numerical experiments to verify the correctness of our results
        and empirically showed that \textbf{alternating projection methods} using the new formula \textbf{converges faster}.
        \item Preprint: \href{https://arxiv.org/abs/2206.11373}
        {Bauschke, Heinz H., Dayou Mao, and Walaa M. Moursi. ``How to project onto the intersection of a closed affine subspace and a hyperplane.'' \textit{arXiv preprint arXiv:2206.11373} (2022)}.
    \end{rSubsection}

\end{rSection}

\begin{rSection}{Projects}

    \begin{rSubsection}{MedTechResolve Student Design Team}{March 2022 - Present}
        {Software Engineering Team Lead}{Waterloo, ON, Canada}
        \item Leading \textbf{computer vision research} on deep learning solutions to \textbf{lung cancer segmentation}.
        \item Leading the \textbf{frontend} and \textbf{backend} teams for developing a triage server, our official website, and an internal human resource management tool.
    \end{rSubsection}

    \begin{rSubsection}{Computer Vision Knowledge Base}{January 2021 - Present}{}{}
        \item Developing a code base for computer vision, including common \textbf{CNN}, \textbf{transformer}, and \textbf{GAN} architectures
        and \textbf{classification}, \textbf{detection}, and \textbf{segmentation} frameworks in TensorFlow.
    \end{rSubsection}

    % \begin{rSubsection}{Stock Trading Bot}{June 2021 - August 2021}{}{}
    %     \item Implemented a stock trading bot based on the \textbf{deep Q-learning} algorithm.
    %     % \item Created a \textbf{data pipeline} to \textbf{scrape} data from yahoo finance and preprocess the data.
    % \end{rSubsection}

    % \begin{rSubsection}{Sentiment Analysis}{September 2020}{}{}
    %     \item Developed classification models on twitter posts labelled  `positive' or `negative'.
    %     \item Preprocessed data by removing \textbf{punctuations} and \textbf{stop words} and \textbf{lemmatizing} words.
    %     \item Experimented and compared the \textbf{naive Bayes}, \textbf{logistic regression}, and \textbf{LSTM} models.
    % \end{rSubsection}

\end{rSection}

\begin{rSection}{Education}

    \begin{rSubsection}{University of Waterloo, Canada}{September 2019 - Present}{}{}
        \item Triple major in \textbf{Computer Science}, \textbf{Statistics}, and \textbf{Optimization} with faculty average 93.46\%.
        % \item Relevant courses: machine learning, computational vision, algorithms, operating systems,
    \end{rSubsection}

\end{rSection}

\end{document}
