\documentclass{resume}
\usepackage[left=0.75in,top=0.6in,right=0.75in,bottom=0.6in]{geometry}
\usepackage{amsmath}

\begin{document}

\begin{rSection}{Technical Skills}

    \begin{itemize}
        \item Languages/Tools: \textbf{Python}, \textbf{C++}, \textbf{CUDA}, \textbf{Java}, SQL \textbar\ Git, Docker, AWS, Apache Kafka, Kubernetes.
        \item ML Libraries: \textbf{NumPy}, \textbf{PyTorch}, \textbf{TensorFlow}, scikit-learn, OpenCV, Matplotlib, Caffe, SciPy.
    \end{itemize}

\end{rSection}

\begin{rSection}{Work Experiences}

    \begin{rSubsection}{Vision and Image Processing Lab}{Jan 2023 \(-\) Present \(\cdot\) 4 mos}
        {Research Assistant - Computer Vision}{Waterloo, ON, Canada}
        \item Research on \textbf{multi-task learning} and identification of hard features.
        \item Research on \textbf{explainable AI} for autonomous vehicles and report to Transport Canada.
        \item Research on copyright detection for \textbf{GAN} and \textbf{diffusion} models.
    \end{rSubsection}

    \begin{rSubsection}{University of Waterloo}{May 2022 \(-\) Aug 2022 \(\cdot\) 4 mos}
        {Research Assistant - Optimization}{Waterloo, ON, Canada}
        \item We proposed a novel formula for projection operations with theoretical proof of correctness and empirical results demonstrating the acceleration it brings to the class of alternating projection algorithms.
        \item Preprint: \href{https://arxiv.org/abs/2206.11373}{Bauschke, Heinz H., Dayou Mao, and Walaa M. Moursi. ``How to project onto the intersection of a closed affine subspace and a hyperplane.'' \textit{arXiv preprint arXiv:2206.11373} (2022)}.
    \end{rSubsection}

    \begin{rSubsection}{NVIDIA Corporation}{Jan 2022 \(-\) Apr 2022 \(\cdot\) 4 mos}
        {Computer Vision Engineer - Autonomous Vehicles}{Santa Clara, CA, United States (Remote)}
        \item Implemented new \textbf{data pipeline} to create clean datasets for model development and comparison.
        \item Enriched \textbf{training pipeline} by implementing and testing more learning rate schedules, sampling mechanisms, and refactoring code for neural network implementation.
        \item Proposed improvements on training config and \textbf{stabilized the training process} and \textbf{reduced training time} from \(\sim\)20h to \(\sim\)3h. Significantly sped up model development.
        \item \textbf{Improved \(\boldsymbol{F_{1}}\)-score} of a traffic light classification model by \(\sim\)\textbf{1\%} on \textbf{end-to-end KPI} test sets by hyperparameter searching from 1000+ experiments.
        \item Debugged memory, latency, and \textbf{performance tests} for multiple classifier nodes on different platforms.
    \end{rSubsection}

    \begin{rSubsection}{MIND Technology, Inc.}{May 2021 \(-\) Aug 2021 \(\cdot\) 4 mos}
        {Machine Learning Engineer - Object Detection}{The Woodlands, TX, United States (Remote)}
        \item Generated \textbf{synthetic data} of lobster pots, human bodies, and mines for \textbf{model pretraining}.
        \item Achieved \textbf{near 1.0 confidence} on synthetic data after fine-tuning the network topology and weights from a \textbf{RetinaNet} trained on MS COCO dataset.
        \item Researched on deployment onto Google Edge TPU with \textbf{TensorFlow Lite} and NVIDIA Jetson Nano with \textbf{TensorRT}, and profiled the usages.
    \end{rSubsection}

\end{rSection}

\begin{rSection}{Projects}

    % \begin{rSubsection}{MedTechResolve Student Design Team}{Mar 2022 - Present \(\cdot\) 11 mos}{}{}
    %     \item \textbf{Team lead} of the computer vision R\&D team on biomedical engineering and HCI related projects.
    % \end{rSubsection}

    \begin{rSubsection}{Computer Vision Code Base
            (\href{https://github.com/danielmao2019/Computer-Vision-PyTorch}{\faGithub PyTorch})
            (\href{https://github.com/danielmao2019/Computer-Vision-TensorFlow}{\faGithub TensorFlow})
        }{Jan 2021 - Present \(\cdot\) 2 yrs 4 mos}{}{}
        \item Production-level implementation of \textbf{data input}, model \textbf{training}, model \textbf{evaluation} pipelines and well-known models for \textbf{image classification}, \textbf{object detection}, and \textbf{semantic segmentation} tasks.
    \end{rSubsection}
    \begin{rSubsection}{Machine Learning Knowledge Base
            \href{https://github.com/danielmao2019/Machine-Learning-Knowledge-Base}{\faGithub}
        }{Jan 2021 - Present \(\cdot\) 2 yrs 4 mos}{}{}
        \item Compilation of papers and notes in machine learning with a focus on \textbf{CNN}, \textbf{Transformer}, \textbf{GAN}, and \textbf{diffusion models}. Other topics include \textbf{multi-task learning}, \textbf{XAI}, and \textbf{NERF}.
    \end{rSubsection}

\end{rSection}

\begin{rSection}{Education}

    \begin{rSubsection}{University of Waterloo, Canada}{Sep 2019 - Present \(\cdot\) 3 yrs 8 mos}{}{}
        \item Triple major in \textbf{Computer Science}, \textbf{Statistics}, and \textbf{Optimization} with faculty average \(\sim\)93\%.
        % \item Relevant courses: machine learning, computational vision, algorithms, operating systems,
    \end{rSubsection}

\end{rSection}

\end{document}
