\documentclass{resume}
\usepackage[left=0.75in,top=0.6in,right=0.75in,bottom=0.6in]{geometry}
\usepackage{amsmath}

\begin{document}

\begin{rSection}{Technical Skills}

    \begin{itemize}
        \item Languages/Tools: \textbf{Python}, \textbf{C++}, \textbf{CUDA}, \textbf{Java}, SQL \textbar\ Git, Docker, AWS, Apache Kafka, Kubernetes.
              % \item Languages/Tools: Python, C++, Java, JavaScript, HTML, CSS, R, SQL.
        \item AI/ML Libraries: \textbf{NumPy}, \textbf{TensorFlow}, \textbf{PyTorch}, scikit-learn, OpenCV, Matplotlib, Caffe, SciPy.
    \end{itemize}

\end{rSection}

\begin{rSection}{Work Experiences}

    \begin{rSubsection}{NVIDIA Corporation}{Jan 2022 \(-\) Apr 2022 \(\cdot\) 4 mos}
        {Computer Vision Engineer - Autonomous Vehicles}{Santa Clara, CA, United States (Remote)}
        \item Created new \textbf{data input pipeline} to define clean datasets for model development and comparison.
        \item Enabled more \textbf{robust training experiments} with more learning rate schedules, sampling mechanisms, and by refactoring model definition.
        \item \textbf{Stabilized the training process} and \textbf{reduced training time} from \(\sim\)20h to \(\sim\)3h to significantly speed up model development.
        \item \textbf{Improved \(\boldsymbol{F_{1}}\)-score} of a traffic light classification model by \(\sim\)\textbf{1\%} on \textbf{end-to-end KPI} test sets by hyperparameter searching from 1000+ experiments.
        \item Debugged memory, latency, and \textbf{performance tests} for multiple classifier nodes on different platforms.
        % and generated performance \textbf{reports} for the AV infra team.
    \end{rSubsection}

    \begin{rSubsection}{MIND Technology, Inc.}{May 2021 \(-\) Aug 2021 \(\cdot\) 4 mos}
        {Machine Learning Engineer - Object Detection}{The Woodlands, TX, United States (Remote)}
        \item Generated \textbf{synthetic data} of lobster pots, human bodies, and mines for \textbf{pretraining} our model.
        \item Achieved \textbf{near 1.0 confidence} on synthetic data after fine-tuning the network topology.
        \item \textbf{Transfer learning} of our RetinaNet model from the COCO 2017 dataset to our domain.
        \item Deployed the model onto Google Edge TPU with \textbf{TensorFlow Lite} and NVIDIA Jetson Nano with\\
        \textbf{TensorRT} and profiled the usages.
    \end{rSubsection}

\end{rSection}

\begin{rSection}{Research Experiences}

    \begin{rSubsection}{Vision and Image Processing Lab}{Jan 2023 \(-\) Present \(\cdot\) 1 mo}
        {Undergraduate Research Assistant}{Waterloo, ON, Canada}
        \item Error analysis by \textbf{variance of gradient} to identify \textbf{hard training examples}.
    \end{rSubsection}

    \begin{rSubsection}{University of Waterloo}{May 2022 \(-\) Aug 2022 \(\cdot\) 4 mos}
        {Undergraduate Research Assistant}{Waterloo, ON, Canada}
        \item Preprint: \href{https://arxiv.org/abs/2206.11373}
        {Bauschke, Heinz H., Dayou Mao, and Walaa M. Moursi. ``How to project onto the intersection of a closed affine subspace and a hyperplane.'' \textit{arXiv preprint arXiv:2206.11373} (2022)}.
        \item Implemented \textbf{numerical experiments} to verify the correctness of our results
        and empirically demonstrated that alternating projection algorithms with the new formula \textbf{converge faster}.
    \end{rSubsection}

\end{rSection}

\begin{rSection}{Projects}

    \begin{rSubsection}{MedTechResolve Student Design Team}{Mar 2022 - Present \(\cdot\) 11 mos}{}{}
        \item \textbf{Team lead} of the computer vision R\&D team on biomedical engineering and HCI related projects.
    \end{rSubsection}

    \begin{rSubsection}{Computer Vision Code Base}{Jan 2021 - Present \(\cdot\) 2 yrs 1 mo}{}{}
        \item \href{https://github.com/danielmao2019/Computer-Vision-TensorFlow}{\faGithub}
        Production-level implementation of \textbf{data input}, model \textbf{training}, and model \textbf{evaluation} pipelines for
        \textbf{image classification}, \textbf{object detection}, and \textbf{semantic segmentation} tasks.
        \item \href{https://github.com/danielmao2019/Machine-Learning-Knowledge-Base}{\faGithub}
        Collection of papers and notes in ML with a focus on \textbf{CNN}, \textbf{Transformer}, and \textbf{GAN} architectures.
    \end{rSubsection}

\end{rSection}

\begin{rSection}{Education}

    \begin{rSubsection}{University of Waterloo, Canada}{Sep 2019 - Present}{}{}
        \item Triple major in \textbf{Computer Science}, \textbf{Statistics}, and \textbf{Optimization} with faculty average 93.46\%.
        % \item Relevant courses: machine learning, computational vision, algorithms, operating systems,
    \end{rSubsection}

\end{rSection}

\end{document}
