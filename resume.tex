\documentclass{resume}
\usepackage[left=0.75in,top=0.6in,right=0.75in,bottom=0.6in]{geometry}
\usepackage{amsmath}

\begin{document}

\begin{rSection}{Technical Skills}

    \begin{itemize}
        \item Languages/Tools: \textbf{Python}, \textbf{C++}, SQL, R \textbar\ Git, Docker, CUDA, AWS, Apache Kafka, Kubernetes.
              % \item Languages/Tools: Python, C++, Java, JavaScript, HTML, CSS, R, SQL.
        \item AI/ML Libraries: \textbf{NumPy}, \textbf{TensorFlow}, \textbf{PyTorch}, scikit-learn, OpenCV, Matplotlib, Caffe, SciPy, pandas.
    \end{itemize}

\end{rSection}

\begin{rSection}{Work Experiences}

    \begin{rSubsection}{NVIDIA Corporation}{January 2022 \(-\) April 2022 \(\cdot\) 4 mos}
        {Computer Vision Engineer - Autonomous Vehicles}{Santa Clara, CA, United States (Remote)}
        \item Defined a new \textbf{data input pipeline} and enabled relevant teams to create clean datasets for model development
        and comparison between different versions.
        \item \textbf{Software implementation} of cyclical learning rate schedules, over/undersampling mechanisms, and \textbf{refactored code} for model definition
        to enable more robust and flexible fine-tuning process.
        \item \textbf{Improved \(\boldsymbol{F_{1}}\)-score} of a traffic light classification model by around \textbf{1\%} on \textbf{end-to-end KPI} test sets
        by fine-tuning from thousands of experiments.
        \item \textbf{Stabilized the training process} and \textbf{reduced training time} from around 20h to around 3h with improved training methodologies.
        \item Fixed memory, latency, and \textbf{performance tests} for multiple classifier nodes on different platforms and generated \textbf{performance reports} for the AV infra team.
    \end{rSubsection}

    \begin{rSubsection}{MIND Technology, Inc.}{May 2021 \(-\) August 2021 \(\cdot\) 4 mos}
        {Machine Learning Engineer - Object Detection}{The Woodlands, TX, United States (Remote)}
        \item Generated \textbf{synthetic data} of lobster pots for \textbf{pretraining} the \textbf{RetinaNet} model.
        \item \textbf{Transferred} a RetinaNet \textbf{object detection} model
        from the COCO 2017 dataset to sonar signals of underwater lobster pots.
        \item Fine-tuned the feature pyramid architecture and achieved \textbf{near 1.0 confidence} on synthetic data.
        \item Deployed the model onto \textbf{Google Edge TPU} using TensorFlow Lite
        and \textbf{NVIDIA Jetson Nano} using TensorRT,
        and profiled the usages.
    \end{rSubsection}

\end{rSection}

\begin{rSection}{Research Experiences}

    \begin{rSubsection}{University of Waterloo}{May 2022 \(-\) August 2022 \(\cdot\) 4 mos}
        {Undergraduate Research Assistant}{Waterloo, ON, Canada}
        \item Preprint: \href{https://arxiv.org/abs/2206.11373}
        {Bauschke, Heinz H., Dayou Mao, and Walaa M. Moursi. ``How to project onto the intersection of a closed affine subspace and a hyperplane.'' \textit{arXiv preprint arXiv:2206.11373} (2022)}.
        \item Proposed and proved a \textbf{closed form formula} for \textbf{projection operations} onto the intersection of a closed affine subspace and a hyperplane in the context of Hilbert spaces.
        \item Implemented \textbf{numerical experiments} to verify the correctness of our results
        and empirically demonstrated that alternating projection algorithms with the new formula \textbf{converge faster}.
    \end{rSubsection}

\end{rSection}

\begin{rSection}{Projects}

    \begin{rSubsection}{MedTechResolve Student Design Team}{March 2022 - Present}{}{}
        \item Leading the \textbf{computer vision R\&D} team on various biomedical engineering projects.
    \end{rSubsection}

    \begin{rSubsection}{Machine Learning Knowledge Base}{January 2021 - Present}{}{}
        \item \href{https://github.com/danielmao2019/Computer-Vision-TensorFlow}{\faGithub}
        Production-level implementation of data input, model training, and model evaluation pipelines for
        \textbf{classification}, \textbf{object detection}, and \textbf{semantic segmentation} tasks.
        \item \href{https://github.com/danielmao2019/Machine-Learning-Knowledge-Base}{\faGithub}
        Collection of papers and notes in ML with a focus on \textbf{CNN}, \textbf{Transformer}, and \textbf{GAN} architectures.
    \end{rSubsection}

\end{rSection}

\begin{rSection}{Education}

    \begin{rSubsection}{University of Waterloo, Canada}{September 2019 - Present}{}{}
        \item Triple major in \textbf{Computer Science}, \textbf{Statistics}, and \textbf{Optimization} with faculty average 93.46\%.
        % \item Relevant courses: machine learning, computational vision, algorithms, operating systems,
    \end{rSubsection}

\end{rSection}

\end{document}
